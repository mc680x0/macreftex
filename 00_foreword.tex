% -*- coding: utf-8 -*-

\chapter{Foreword}

\begin{center}
\textit{``Hello, I am Macintosh. It sure is great to get out of that bag!''}
\end{center}

\paragraph{%
On January the 24th of 1984, Macintosh introduced itself to the world. In comparison %
to the dolled-up computers we take for granted today, personal computers of the late %
1970s and early 1980s were ... rather impersonal. Personal meant that a person might %
own one, not that a person may be comfortable using one. Macintosh represented a fundamental %
shift in how people interacted with machines. \\
\textit{Instead of teaching humans about computers,} the theory went, %
\textit{we should teach computers about humans.} A laudable goal indeed, at a % 
time when most peoples' knowledge of computers was either arcade-style gaming or %
the necessity of memorising an arcane set of textual commands for getting work done. %
}

\begin{center}
\textit{``Unaccustomed as I am to public speaking, I'd like to share with you\\  a %
maxim I thought of the first time I met an IBM mainframe: \\Never trust a computer %
that you can't lift!''}
\end{center}

\paragraph{%
Macintosh was a computer for the masses. Every aspect of it was engineered to be %
appealing to the everyday user. Macintosh looked radically different from other %
personal computers of the time, many of which were shaped like suitcases or overly %
thick keyboards. Macintosh's price, too, was designed to be appealing - in comparison %
to the Lisa, with which it shared the 68000 at its heart and the idea of an entirely %
graphical interface, Macintosh was the very epitome of value, at \$2,495 (the Lisa cost %
\$9,995 at its inception, and at the time of Macintosh's introduction was still \$3,495 %
for a base model, a full thousand dollars more expensive than Macintosh.) %
}

\paragraph{%
Many books have been written about the historical context of Macintosh, and the line %
that dutifully followed it; hence, there is no need to go into greater detail here. %
This book exists to fulfill a need, and to fill a niche. There is a thriving community %
of people from many walks of life, from well-to-do collectors to impoverished students, %
united by a love of Macs. Unfortunately, many of the older Macs they so dearly love are %
rather past their prime; so much so that modular repairs are sometimes no longer sufficient %
to keep machines running. Component-level repair is becoming an unpleasant fact of life for %
those who would keep their Macintoshes in workable condition; among the prime culprits are %
aged capacitors and blown resistor networks. %
}

\paragraph{%
Of course, in repairing Macintoshes it helps to know something about them - how they should %
behave when working to specifications, and it especially helps to know what their limits and %
quirks are. For instance, while it is physically possible to install 32MB of RAM in a Colour %
Classic, no more than 10MB would be seen, and thus it would be a waste of (now fairly expensive) %
RAM SIMMs. %
}

\paragraph{%
This need is met, for users of Mac OS X and iOS, by a piece of software called Mactracker. It is also %
available for Windows, but has not been updated on that platform for some years. Formerly, it was %
available for Mac OS 8.6; this version has been withdrawn. A Linux port has been enquired about, %
and as of the time of writing, it has not materialised and is unlikely to. %
Ian Page has done a great good in making Mactracker available to the community at no %
charge, and it is a supremely useful resource which I recommend highly to anyone who works on Macs %
for their livelihood, or in their leisure. I would be lying to say that it has not been %
incredibly useful for researching the bulk of specifications in this tome. Thankfully %
I can get away with using the Windows version, outdated though it be, due to only covering %
Old World Macs. %
}

\paragraph{%
This book, however, seeks not only to provide the practical specifications of known Macintoshes, %
but also to collect, in one place, known repair techniques and background technical information. %
}