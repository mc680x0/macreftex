% -*- coding: utf-8 -*-

\part{PowerPC Macintoshes}

\chapter{The \textsl{\textbf{\textrm{PowerPC}}\texttrademark}}

\paragraph{%
Introduced in Early 1994 with the x100 series Power Macintosh, the PowerPC was %
the beating heart of the Macintosh franchise until the middle of 2006, going %
through five generations of hardware made by Apple, and several other machines %
made by licenced Clone manufacturers such as UMAX, Radius, Daystar, and %
Motorola. While PowerPC-based machines did not completely displace their 68k %
forebears until October of 1996, with the discontinuation of the PowerBook 190cs, %
they quickly established themselves as Apple's mainstay. %
}

\paragraph{%
The transition from 68k to PowerPC was not without controversy, nor were the %
first-generation PowerPC machines without their foibles - in fact, it could be %
said that foibles and odd behaviours were a defining part of the Mac experience, %
if ever you had to spend a great deal of time working on Mac hardware or software%
... %
these foibles and unique Mac-only issues would continue well into the salad years %
of the Power Macintosh line. %
}

\paragraph{%
Here we concern ourselves only with the so-called ``Old World'' Power Macintoshes, %
that is to say any and all Power Macintoshes that look like Computers, as opposed %
to pieces of fruit. In practical terms, if you're looking for the specifics of a %
Power Macintosh G3 (Blue and White), another source is your best bet; but we'll %
be your faithful guide to anything up to and including the Power Macintosh G3 %
(tower), the ``Outrigger'' style Power Macintosh G3 (Desktop), and the infamous %
``Molar Mac'' Power Macintosh G3 (All-in-One). %
}
